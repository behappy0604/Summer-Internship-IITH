\documentclass[journal,12pt,twocolumn]{IEEEtran}
\usepackage{setspace}
\usepackage{gensymb}
\singlespacing
\usepackage[cmex10]{amsmath}
\usepackage{amsthm}
\usepackage{mathrsfs}
\usepackage{txfonts}
\usepackage{stfloats}
\usepackage{bm}
\usepackage{cite}
\usepackage{cases}
\usepackage{subfig}
\usepackage{longtable}
\usepackage{multirow}
\usepackage{enumitem}
\usepackage{mathtools}
\usepackage{steinmetz}
\usepackage{tikz}
\usepackage{circuitikz}
\usepackage{verbatim}
\usepackage{tfrupee}
\usepackage[breaklinks=true]{hyperref}
\usepackage{graphicx}
\usepackage{tkz-euclide}
\usetikzlibrary{calc,math}
\usepackage{listings}
    \usepackage{color}                                            %%
    \usepackage{array}                                            %%
    \usepackage{longtable}                                        %%
    \usepackage{calc}                                             %%
    \usepackage{multirow}                                         %%
    \usepackage{hhline}                                           %%
    \usepackage{ifthen}                                           %%
    \usepackage{lscape}     
\usepackage{multicol}
\usepackage{chngcntr}
\DeclareMathOperator*{\Res}{Res}
\newcommand{\myvec}[1]{\ensuremath{\begin{pmatrix}#1\end{pmatrix}}}
\renewcommand\thesection{\arabic{section}}
\renewcommand\thesubsection{\thesection.\arabic{subsection}}
\renewcommand\thesubsubsection{\thesubsection.\arabic{subsubsection}}
\renewcommand\thesectiondis{\arabic{section}}
\renewcommand\thesubsectiondis{\thesectiondis.\arabic{subsection}}
\renewcommand\thesubsubsectiondis{\thesubsectiondis.\arabic{subsubsection}}
\hyphenation{op-tical net-works semi-conduc-tor}
\def\inputGnumericTable{}                                 %%
\lstset{
%language=C,
frame=single, 
breaklines=true,
columns=fullflexible
}
\begin{document}
\newtheorem{theorem}{Theorem}[section]
\newtheorem{problem}{Problem}
\newtheorem{proposition}{Proposition}[section]
\newtheorem{lemma}{Lemma}[section]
\newtheorem{corollary}[theorem]{Corollary}
\newtheorem{example}{Example}[section]
\newtheorem{definition}[problem]{Definition}
\newcommand{\BEQA}{\begin{eqnarray}}
\newcommand{\EEQA}{\end{eqnarray}}
\newcommand{\define}{\stackrel{\triangle}{=}}
\bibliographystyle{IEEEtran}
\providecommand{\mbf}{\mathbf}
\providecommand{\pr}[1]{\ensuremath{\Pr\left(#1\right)}}
\providecommand{\qfunc}[1]{\ensuremath{Q\left(#1\right)}}
\providecommand{\sbrak}[1]{\ensuremath{{}\left[#1\right]}}
\providecommand{\lsbrak}[1]{\ensuremath{{}\left[#1\right.}}
\providecommand{\rsbrak}[1]{\ensuremath{{}\left.#1\right]}}
\providecommand{\brak}[1]{\ensuremath{\left(#1\right)}}
\providecommand{\lbrak}[1]{\ensuremath{\left(#1\right.}}
\providecommand{\rbrak}[1]{\ensuremath{\left.#1\right)}}
\providecommand{\cbrak}[1]{\ensuremath{\left\{#1\right\}}}
\providecommand{\lcbrak}[1]{\ensuremath{\left\{#1\right.}}
\providecommand{\rcbrak}[1]{\ensuremath{\left.#1\right\}}}
\theoremstyle{remark}
\newtheorem{rem}{Remark}
\newcommand{\sgn}{\mathop{\mathrm{sgn}}}
\providecommand{\abs}[1]{\vert#1\vert}
\providecommand{\res}[1]{\Res\displaylimits_{#1}} 
\providecommand{\norm}[1]{\Vert#1\rVert}
%\providecommand{\norm}[1]{\lVert#1\rVert}
\providecommand{\mtx}[1]{\mathbf{#1}}
\providecommand{\mean}[1]{E[ #1 ]}
\providecommand{\fourier}{\overset{\mathcal{F}}{ \rightleftharpoons}}
%\providecommand{\hilbert}{\overset{\mathcal{H}}{ \rightleftharpoons}}
\providecommand{\system}{\overset{\mathcal{H}}{ \longleftrightarrow}}
	%\newcommand{\solution}[2]{\textbf{Solution:}{#1}}
\newcommand{\solution}{\noindent \textbf{Solution: }}
\newcommand{\cosec}{\,\text{cosec}\,}
\providecommand{\dec}[2]{\ensuremath{\overset{#1}{\underset{#2}{\gtrless}}}}
\newcommand{\myvec}[1]{\ensuremath{\begin{pmatrix}#1\end{pmatrix}}}
\newcommand{\mydet}[1]{\ensuremath{\begin{vmatrix}#1\end{vmatrix}}}
\numberwithin{equation}{subsection}
\makeatletter
\@addtoreset{figure}{problem}
\makeatother
\let\StandardTheFigure\thefigure
\let\vec\mathbf
\renewcommand{\thefigure}{\theproblem}
\def\putbox#1#2#3{\makebox[0in][l]{\makebox[#1][l]{}\raisebox{\baselineskip}[0in][0in]{\raisebox{#2}[0in][0in]{#3}}}}
     \def\rightbox#1{\makebox[0in][r]{#1}}
     \def\centbox#1{\makebox[0in]{#1}}
     \def\topbox#1{\raisebox{-\baselineskip}[0in][0in]{#1}}
     \def\midbox#1{\raisebox{-0.5\baselineskip}[0in][0in]{#1}}
\vspace{3cm}
\title{ASSIGNMENT-9}
\author{Ojaswa Pandey}
\maketitle
\newpage
\bigskip
\renewcommand{\thefigure}{\theenumi}
\renewcommand{\thetable}{\theenumi}
Download all python codes from 
\begin{lstlisting}
https://github.com/behappy0604/Summer-Internship-IITH/tree/main/Assignment-9
\end{lstlisting}
%
and latex-tikz codes from 
%
\begin{lstlisting}
https://github.com/behappy0604/Summer-Internship-IITH/tree/main/Assignment-9
\end{lstlisting}
%
\section{Question No. 8.1} 
\item Let U and V be two independent zero mean Gaussian random variables of variances $\dfrac{1}{4}$ and $\dfrac{1}{9}$ respectively. The probability $P(3V\geqslant2U)$ is
\begin{enumerate}
\begin{multicols}{4}
\setlength\itemsep{2em}
\item $
\dfrac{4}{9}
$
\item $
\dfrac{1}{2}
$
\item $
\dfrac{2}{3}
$
\item $
\dfrac{5}{9}
$
\end{multicols}
\section{Solution}

Since $U$ and $V$ are given to be normal random variables, therefore their difference will also be a normal random variable.\\

Here, let 
\begin{align}
   X=3V-2U 
\end{align}
 \\where $X$ is also a normal random variable with mean given as
 \begin{align}
  \mathbb{E}[X_M]=\mathbb{E}[3V-2U]
 \end{align}
 \begin{align}
     \mathbb{E}[X_M]=\mathbb{E}[3V]-\mathbb{E}[2U]
 \end{align}
 \begin{align}
     \mathbb{E}[X_M]=3\mathbb{E}[V]-2\mathbb{E}[U]
 \end{align}
 \begin{align}
     \mathbb{E}[X_M]=0
 \end{align}
 and variance

\begin{align}
    \implies \mathbb{E}[X^2]-\mathbb{E}[X]^2
\end{align}\\
\begin{align}
    \implies \mathbb{E}[(3V-2U)^2]-(3\mathbb{E}[V]-2\mathbb{E}[U])^2
\end{align}
\begin{multline}
\implies \mathbb{E}[9V^2+4U^2-12UV]-9\mathbb{E}[V]^2-4\mathbb{E}[U]^2+\\12\mathbb{E}[V]\mathbb{E}[U] 
\end{multline}
\begin{multline}
\implies 9\mathbb{E}[V^2]+4\mathbb{E}[U^2]-12\mathbb{E}[UV]-9\mathbb{E}[V]^2-4\mathbb{E}[U]^2+\\12\mathbb{E}[V]\mathbb{E}[U]
\end{multline}
\begin{multline}
 \implies 9(\mathbb{E}[V^2]-\mathbb{E}[V]^2)+4(\mathbb{E}[U^2]-\mathbb{E}[U]^2)-\\12\mathbb{E}[UV]-\mathbb{E}[U]\mathbb{E}[V])
\end{multline}
\begin{align}
    \implies 9($variance of V$)+4($variance of U$)-12(0)
    \end{align}
    \\ (Since $\mathbb{E}[UV]$=$\mathbb{E}[U]$ $\mathbb{E}[V]$ for independent random variable)

\begin{align}
    \implies 9\times \frac{1}{9}+4\times \frac{1}{4}
\end{align}
\begin{align}
    \implies 2
\end{align}
  
\begin{lemma}
The area under the Gaussian PDF curve below and above the mean value is $\frac{1}{2}$
\begin{align}
    \implies  P(X>=X_M)= {\frac{1}{2}}   
\end{align}
The area under the curve and the x-axis is unity.\\
\end{lemma}
So it will be symmetric about mean that is 0.
\begin{align}
    \therefore P(X>=0)= \boxed{\frac{1}{2}}    $(by symmetry property)$
\end{align}
Cumulative density function of the curve 
\begin{align}
    CDF = \int_{-\infty}^{x}f(t)\,dt= \frac{1}{2}
\end{align}
Q-function
\begin{align}
Q(X)= 1-CDF= 1-\frac{1}{2}= \frac{1}{2}
\end{align}
Hence option (b) is correct.
\end{enumerate}
\end{document}
